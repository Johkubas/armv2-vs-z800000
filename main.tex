\documentclass[11pt,a4paper]{article}

\usepackage[utf8]{inputenc}
\usepackage[T1]{fontenc}
\usepackage[lithuanian,english]{babel}
\usepackage{csquotes}
\usepackage{geometry}
\geometry{margin=2.5cm}
\usepackage{hyperref}
\usepackage{setspace}
\onehalfspacing

% literatūrai vėliau galėsim naudoti biblatex arba natbib;
% kol kas pakaks paprastų (Author, Year) nuorodų tekste.

\title{ARMv2 vs. Zilog Z80000:\\
Kompiuterių architektūrų palyginimas}
\author{Vardas Pavardė}
\date{\today}

\begin{document}

\maketitle

\section{Įvadas}
ARMv2 ir Z80000 architektūros, abi ~1986–1987 laikotarpio. 

\section{Elementinė bazė ir fizinės savybės}

\subsection{ARMv2}

ARMv2 (konkrečiai ARM2 procesorius) buvo pristatytas apie 1987 metus ir
gaminamas naudojant CMOS technologiją su maždaug 2~µm gamybos procesu.
Šis procesorius turėjo apie 25 tūkst. tranzistorių, kas tuo metu buvo itin
mažas skaičius 32 bitų architektūrai. Dėl mažo tranzistorių kiekio ARM2
pasižymėjo labai mažu energijos suvartojimu ir nedideliu šilumos išsiskyrimu.

Tipinis taktinis dažnis siekė apie 8~MHz. Procesorius buvo naudojamas
\emph{Acorn Archimedes} kompiuteriuose. Fizinė procesoriaus realizacija buvo
monokristalinis mikroprocesorius viename luste. Sistema pasižymėjo nedideliais
gabaritais, maža mase ir itin geru energijos efektyvumu, lyginant su kitais
to meto 32 bitų sprendimais.

\subsection{Zilog Z80000}

Zilog Z80000 buvo pristatytas apie 1986 metus kaip 32 bitų CISC tipo
mikroprocesorius. Jis buvo gaminamas naudojant VLSI technologiją ir turėjo apie
91 tūkst. tranzistorių. Palyginus su ARMv2, Z80000 buvo gerokai sudėtingesnis
ir didesnis lustas.

Procesorius palaikė iki 4~GiB adresuojamos atminties erdvę, turėjo vidinį
atminties valdymo bloką (MMU) ir šešių pakopų komandų vykdymo vamzdyną
(\emph{pipeline}). Tipiniai taktų dažniai siekė apie 10--20~MHz.

Z80000 sistemos buvo orientuotos į galingesnes darbo stotis ir pramonines
sistemas. Dėl didesnio tranzistorių skaičiaus šis procesorius naudojo daugiau
energijos, turėjo didesnį šilumos išskyrimą ir reikalavo sudėtingesnės
aušinimo bei maitinimo sistemos.

\subsection{Palyginimas}

ARMv2 pasižymėjo itin kompaktišku dizainu, mažu tranzistorių skaičiumi,
mažu energijos suvartojimu ir paprasta fizine realizacija, todėl puikiai
tiko kompaktiškoms sistemoms.

Tuo tarpu Zilog Z80000 buvo didesnio našumo, bet ir gerokai sudėtingesnis
procesorius su didesniu energijos poreikiu, orientuotas į galingas
skaičiavimo sistemas. Abu procesoriai naudojo VLSI technologijas, tačiau
jų projektavimo filosofija buvo iš esmės skirtinga.

\section{Architektūros tipas}
% Atsakymas į 3 klausimą – ar tai registrinė, RISC/CISC, ir t. t.

% ... vėliau pridėsim daugiau sekcijų pagal 4–20 klausimus.

\end{document}
