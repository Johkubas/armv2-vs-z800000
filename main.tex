\documentclass[11pt,a4paper]{article}

\usepackage[utf8]{inputenc}
\usepackage[T1]{fontenc}
\usepackage[lithuanian,english]{babel}
\usepackage{csquotes}
\usepackage{geometry}
\geometry{margin=2.5cm}
\usepackage{hyperref}
\usepackage{setspace}
\onehalfspacing

% literatūrai vėliau galėsim naudoti biblatex arba natbib;
% kol kas pakaks paprastų (Author, Year) nuorodų tekste.

\title{ARMv2 vs. Zilog Z80000:\\
Kompiuterių architektūrų palyginimas}
\author{Vardas Pavardė}
\date{\today}

\begin{document}

\maketitle

\section{Įvadas}
Trumpai pristatyk, kodėl pasirinktos ARMv2 ir Z80000 architektūros
(ir kad jos abi iš ~1986–1987 laikotarpio).

\section{Elementinė bazė ir fizinės savybės}
% Čia bus atsakymas į 2 klausimą.
\subsection{ARMv2}
% Aprašysim: CMOS, 2 µm procesas, 8 MHz, labai mažas
% tranzistorių skaičius, maža galia, naudota Acorn Archimedes ir t.t.

\subsection{Zilog Z80000}
% Aprašysim: 32 bitų procesorius, ~1986 m., ~91 000 tranzistorių, 
% palaikoma iki 4 GiB adresavimo, didesnė kompleksiškumas ir pan.

\section{Architektūros tipas}
% Atsakymas į 3 klausimą – ar tai registrinė, RISC/CISC, ir t. t.

% ... vėliau pridėsim daugiau sekcijų pagal 4–20 klausimus.

\end{document}
